%% to produce a PDF copy, issue the following command:
%%
%%     pdflatex propositional-logic-examples.tex
%%
%% in the same directory containing the LaTeX style files:
%%
%%     prooftree.sty  and  boxproof.sty

\documentclass[11pt,leqno,fleqn]{article}

\usepackage{amsfonts}
\usepackage{amsmath}

\usepackage{graphicx} 
\usepackage{times}              % better fonts for mathematical symbols
\usepackage{bm}                 % unlike \boldmath,
                                % \bm can be used anywhere within math mode
\usepackage[scaled=0.9]{helvet} % makes text a little smaller throughout,
                                % but not the text in math mode.
\usepackage{./tex/latex/misc/prooftree}
\usepackage{./tex/latex/misc/boxproof}

\setlength\hoffset{-5pt}      % horizontal offset, to move text horizontally
\setlength{\textwidth}{4.5in} % try different widths
\setlength\voffset{-5pt}      % vertical offset, to move text vertically
\setlength{\textheight}{7in}  % try different heights

\newcommand{\Hide}[1]{}             % use \Hide{bla bla} to hide ``bla bla''
\newcommand{\code}[1]{\texttt{#1}}  % use \code{...} to produce ASCII chars
\newcommand{\Intro}[1]{{#1}{\textrm{i}}}
\newcommand{\Elim}[1]{{#1}{\textrm{e}}}



\title{CS 512, Spring 2014
       \\[1ex]
       \textbf{Assignment 2}}
\author{Shan Sikdar} 
\date{Due Monday January 27th} % omit date

\begin{document}

\maketitle

\section{Problem 1 }

Algebraic Structure: $\tau = (T, Lt,  \mathbb{N}, Rt, height) $. Define the size operation $|| : \tau \to \mathbb{N} $ and the height operation height: $\tau \to \mathbb{N}$ in the style of the definitions for the operations Node Lt, and Rt.
 If $ t \in T$ is represented graphically by a finite binary tree, then $|t|$ should return should return the total number of leaf nodes in t + the total number of internal nodes in t, and the hieght(t) should return the length of the longest path in t.
\\

 \begin{equation}
  |t|=\begin{cases}
    |t_1| + |t_2|, & \text{if  }  t = <t_1  \ \ t_2>.\\
    1, & \text{otherwise}
  \end{cases}
\end{equation}

 \begin{equation}
  height(x) =\begin{cases}
    max(|t_1| + |t_2|) + 1, & \text{if  }  t = <t_1  \ \ t_2>.\\
    0, & \text{otherwise}
  \end{cases}
\end{equation}

\newpage

\section{Problem 2}

Algebraic Structure: $ A = (\mathbb{N}-{0},lcm,gcd, \trianglelefteq)$ where for all $m,n \in \mathbb{N}-{0}$, we have that: $m \trianglelefteq n$ iff ''m divides n''.\\
(a) Show that A is a lattice, where lcm playes the role of $\lor$ and gcd plays the role of $\land$.\\
(b) Is A a distributice lattice? Justify your answer carefully, based on the distributivity axioms that A must satisfy.\\ 
\\
(a)\\
Need to Show:\\
(1) Show $ (A, \trianglelefteq)$ is a poset.\\

For all $ a,b,c \in \mathbb{N}-{0}$\\
 \textbf{reflexive:}  ''a divides a'' therefore $  a \trianglelefteq a $\\
 \textbf{anit-symmetric:} \\
Assume  $a \trianglelefteq b$ and $b \trianglelefteq a$. Then ''a divides b'' and ''b divides a''. \\
??\\
  \\
\textbf{transitive:}\\
Assume $a \trianglelefteq b$ and $b \trianglelefteq c$. Then ''a divides b'' and 'b divdes c''.  \\
If ''a divides b'' then b must be some multiple of a ( $ b = m \cdot  a, m \in \mathbb{N}-{0}$).\\ 
If ''b divides c'' then c must be some multiple of b ( $ c = n \cdot  b, m \in \mathbb{N}-{0}$).\\ 
So $c = n \cdot b = n \cdot (m \cdot a) $ where $m,n \in \mathbb{N}-{0}$\\
But then c is some multiple of a. So ''a divides c''.\\
Therefore $ a \trianglelefteq c$.\\
\\
(2) For all $m,n \in A$ the least upper bound of a and b in the ordering $\trianglelefteq$ exists, is unique m and is the result of the operation $m \lor n$\\
\\
\\
\\
(3) For all $m,n \in A$ the greatest lower bound of m and n in the ordering $\trianglelefteq$ exists, is unique m and is the result of the operation $m \land n$\\
% the number 1 is the lowest number element in $\mathbb{N}-{0}$. Also note that for any element n, $1 \tringlelefteq n $. So '' 1 divides any element n''. So 1 exisits in the ordering.  Now since gcd is the greatest number that divides both n, and m. If n and m are relatively prime 
% 
% Uniqueness: Assume not then there exits an x and 1 that are both the greatest lower bound. But then since both are in the natural numbers x <= n n < = n ==> x = n
%

\section{Problem 3: Excercise 1.5.3 parts (a),(b),(c)}

(a) Show that $\{ \neg{},\land \},\{ \neg{},\to \},\{\to,\bot \}$ are adequate sets of connectives for propositional logic.\\

(a1)   $\{ \neg{},\land \}$:\\
If I can show that there is an expression that is equivalent to $ p \lor q$, then this will also show that there exists an equivelent for $\to$ Since from the example we know $p \to q \equiv \neg{p} \lor q $. In order to do that I need to find an expression that has same truth values as $p \lor q$.
Using a truth table:\\


\end{document}

