%% to produce a PDF copy, issue the following command:
%%
%%     pdflatex propositional-logic-examples.tex
%%
%% in the same directory containing the LaTeX style files:
%%
%%     prooftree.sty  and  boxproof.sty

\documentclass[11pt,leqno,fleqn]{article}

\usepackage{amsfonts}
\usepackage{amsmath}
\usepackage{qtree}

\usepackage{graphicx} 
\usepackage{times}              % better fonts for mathematical symbols
\usepackage{bm}                 % unlike \boldmath,
                                % \bm can be used anywhere within math mode
\usepackage[scaled=0.9]{helvet} % makes text a little smaller throughout,
                                % but not the text in math mode.
\usepackage{./tex/latex/misc/prooftree}
\usepackage{./tex/latex/misc/boxproof}


\usepackage{amssymb}
\usepackage{fullpage, enumerate}
\usepackage[usenames]{color}
\usepackage{graphicx, subfig}


\setlength\hoffset{-5pt}      % horizontal offset, to move text horizontally
\setlength{\textwidth}{4.5in} % try different widths
\setlength\voffset{-5pt}      % vertical offset, to move text vertically
\setlength{\textheight}{7in}  % try different heights

\newcommand{\Hide}[1]{}             % use \Hide{bla bla} to hide ``bla bla''
\newcommand{\code}[1]{\texttt{#1}}  % use \code{...} to produce ASCII chars
\newcommand{\Intro}[1]{{#1}{\textrm{i}}}
\newcommand{\Elim}[1]{{#1}{\textrm{e}}}

\newcommand{\TTT}{\bm{\mathsf{T}}}
\newcommand{\FFF}{\bm{\mathsf{F}}}

\title{CS 512, Spring 2014
       \\[1ex]
       \textbf{Assignment 4}}
\author{} 
\date{} % omit date

\begin{document}

\maketitle

\section{Problem 1: 2.3.11 a }

 $P(b) \vdash\forall x (x  = b \to P(x))$

\begin{proofbox}
   \label{d1}\: P(b) \= \textrm{premise} \\
   \[
      \label{d2}\: x_0  \= \textrm{fresh} \\
      \[
         \label{d3}\: x_0 = b    \= \textrm{assume} \\
         \label{d4}\: P(x_0) \= \Elim = \ \ref{d1},\ref{d3}\\
      \]
      \:(x_0 = b) \to P(x_0) \= \Intro\to \ref{d3},\ref{d4} \\
   \]
   \: \forall x ((x = b) \to P(x)) \= \Intro {\forall x } \ \  2-5 \\
\end{proofbox}

\newpage
\section{Problem 2: 2.3.12 }

 $S \to \forall x Q(x) \vdash \forall x (S \to Q(x))$

\begin{proofbox}
   \label{d1}\: S \to \forall x Q(x) \= \textrm{premise} \\
   \[
      \label{d2}\: x_0  \= \textrm{fresh} \\
      \[
         \label{d3}\:S    \= \textrm{assume} \\
         \label{d4}\: \forall x Q(x) \= \Elim \to \ \ref{d1},\ref{d3}\\
         \label{d5}\:  Q(x_0) \= \Elim {\forall x \ } \  \ \\ref{d4}\\
      \]
      \: S \to Q(x_0) \= \Intro\to \ref{d3},\ref{d5} \\
   \]
   \: \forall x (S \to Q(x)) \= \Intro {\forall x } \ \  2-6 \\
\end{proofbox}




\newpage
\section{Problem 3: 2.3.13(a) }

 $\forall x P(a,x,x), \forall x \forall y \forall z (P(x,y,z) \to P(f(x), y, f(z)) ) \vdash P(f(a),a,f(a)) $

\begin{proofbox}
   \label{d1}\: \forall x P(a,x,x) \= \textrm{premise} \\
   \label{d2}\: \forall x \forall y \forall z (P(x,y,z) \to P(f(x), y, f(z)) ) \= \textrm{premise} \\  
    \label{d3}\:  P(a,a,a) \= \Elim {\forall x \ } 1\\
    \label{d4}\: \forall y \forall z (P(a,y,z) \to P(f(a),y,f(z))) \= \Elim  {\forall x \ } 2\\
     \label{d5}\: \forall z (P(a,a,z) \to P(f(a),a,f(z))) \= \Elim  {\forall y \ } 4\\
       \label{d5}\: P(a,a,a) \to P(f(a),a,f(a)) \= \Elim  {\forall z \ } 5\\
         \label{d5}\: P(f(a),a,f(a)) \= \Elim \to 3,6\\
\end{proofbox}


\section{Problem 3: 2.3.13(b) }

 $\forall x P(a,x,x), \forall x \forall y \forall z (P(x,y,z) \to P(f(x), y, f(z)) ) \vdash \exists z P(f(a),z,f(f(a))) $

\begin{proofbox}
   \label{d1}\: \forall x P(a,x,x) \= \textrm{premise} \\
   \label{d2}\: \forall x \forall y \forall z (P(x,y,z) \to P(f(x), y, f(z)) ) \= \textrm{premise} \\  
    \label{d3}\:  P(a,f(a),f(a)) \= \Elim {\forall x \ } 1\\
    \label{d4}\: \forall y \forall z (P(a,y,z) \to P(f(a),y,f(z))) \= \Elim  {\forall x \ } 2\\
     \label{d5}\: \forall z (P(a,f(a),z) \to P(f(a),f(a),f(z))) \= \Elim  {\forall y \ } 4\\
       \label{d5}\: P(a,f(a),f(a)) \to P(f(a),a,f(f(a))) \= \Elim  {\forall z \ } 5\\
         \label{d5}\: P(f(a),f(a),f(f(a))) \= \Elim \to 3,6\\
          \label{d5}\: \exists z P(f(a),z,f(f(a))) \= \Intro {\exists z \ } 7 \\        
\end{proofbox}

\newpage
\section{Problem 4}

%problem 4
\begin{enumerate}[(a)]
\item
The formula $\forall x \forall y \exists z (R(x,y) \to R(y,z))$ is not true in the model $M$. Eg, we have $(b,a) \in R^M$, but there is no $m \in A$ with $(a,m) \in R^M$ contrary to what the formula claims. Thus, we may defeat this formula by choosing $b$ for $x$ and $a$ for $y$ to construct the contradiction to the truth of this formula over the given model.

\item
The formula $\forall x \forall y \exists x(R(x,y) \to R(y,z))$ is true in the model $M$. We may list all the elements of $R$ in a cyclic way as $(a,b), (b,c), (c,b)$, the cycle being the last two pairs. Thus, for any choice of $x$ and $y$ we can find some $z$ so that the implication $R(x,y) \to R(y,z)$ is true.

\end{enumerate}

\section{Problem 5}

%priblem 5
\begin{enumerate}[(i)]
\item
We choose a model with $A$ being the set of integers. We define $(n,m) \in P^M$ if and only if, $n$ is less than or equal to $m$ i.e $(n \leq m)$. Evidently, this interpretation of $P$ is reflexive (every integer is less than equal to itself) and transitive: $n \leq m$ and $m \leq k$ implies $n \leq k$. However, 2 $\leq$ 3 and 3 $\nleq$ 2 show that this interpretation cannot be symmetric.

\item
We choose as set $A$ the sons of a gentleman. We interpret $P(x,y)$ as "$x$ is brother of $y$". Clearly, this relation is transitive and symmetric but not reflexive.

\item
We define $A = {a,b,c}$ and $P^M = \{(a,a),(b,b),(c,c),(a,c),(a,b),(b,a),(c,a)\}$. \\
Note that this interpretation is reflexive and symmetric. We also have that $(b,a)$ and $(a,c)$ are in $P^M$. Thus, we would need $(b,c) \in P^M$ to secure transitivity of $P^M$. Since this is not the case, we infer that this interpretation of $P$ is not transitive.
\end{enumerate}



\newpage
\section{Problem 6}
\begin{enumerate}[(1)]
\item %number 1
What does Lemma 'one' says? Express it in English precisely. Is it satisfiable, unsatisfiable, or valid?\\
\\
Lemma 1 states:\\
 If:\\
- R contains an edge between every pair of nodes.\\
-  S contains at most one edge between every node\\
- R is contained in S\\
\\
Then all three of these things imply: S is contained in R.\\
\\
This lemma is valid.
\item %number 2
Is Lemma 'three' equivalent to Lemma 'one' and Lemma 'two' above ? Express it in English precisely. If not, is it satisfiable, unsatisfiable,  or valid ?\\
\\
Lemma 3 states:\\
If:\\
- R contains an edge between every pair of nodes. \\
-  S contains at most one edge between every node \\
- R is contained in S. \\
\\
Then all three of these things imply: S = R.\\
\\
Since for S = R, S must be contained in R, Lemma 3 is equivelent to Lemma 1 and 2.\\
This lemma is valid.\\

\newpage
\item %number 3
Is Lemma 'four' equivalent to Lemmas 'one', 'two' or 'three' above? Justify your answer carefully in English. If not, is it satisfiable, unsatisfiable, or valid?\\
Lemma 4 states:\\
If:\\
- R contains an edge between every pair of nodes. \\
- S has no edges. \\
- R is contained in S. \\ 
Then all three of these things imply: S = R.\\
\\
Lemma four is not equivalent to lemmas 1,2,3 because lemma 4 states that if S(x, y) then x = y which is a more strict assertion than in lemmas 1,2,3.
This lemma is valid.\\

\item %number 4
Is Lemma 'five' equivalent to 'one', 'two', 'three', or 'four' above? Justify your answer carefully in English. If not, is it satisfiable, unsatisfiable, or valid?\\
\\
Lemma 5 specifies  that there exists three distinct nodes, while the previous lemma's do not mention how many nodes there need to be. Therefore Lemma 5 is not equivelent to Lemmas 1,2,3,4.  Also, the conclusion in lemma five is that S is not equal to R, which is not what lemmas one through three conclude. 
\\
The lemma is satisfiable.

\item  %number 5
Is Lemma 'six' equivalent to 'one', 'two', 'three', 'four', or 'five'? Justify your answer carefully in English. If not, is it satisfiable, unsatisfiable, or valid ?\\
Lemma 6 is not equivelent to lemmas 1,2,3 because in lemma 1,2,3 we have that for a node x, there must  be R(x,x), but in lemma 6 this is not allowed since x = x.\\
Lemma 6 is not equivelent to lemma 4 since in lemma 6 we could have S(x,y) where x and y are different, but according to lemma 4 S(x,y) implies that x and y are the same.\\
Lemma 6 is not equivelent to lemma 5 since it does not specifiy that there has to exist three distinct nodes.\\
Lemma is satisfiable.\\

\end{enumerate}

\section{Problem 7}
%problem 6
\begin{enumerate}[(i)]
\item
``there exists exactly three values of x which make $\varphi(x)$ true" \\
$\exists a \exists b \exists c (\varphi(a) \land \varphi(b) \land \varphi(c) \land \neg(a = b) \land \neg(b = c) \land \neg(a = c) \land \forall x(\neg (x = a) \land \neg(x = b) \land \neg(x = c)) \to \neg \varphi(x))$
\\
\item
``there exists at least three values of x which makes $\varphi(x)$ true" \\
$\exists a \exists b \exists c (\varphi (a) \land \varphi(b) \land \varphi(c) \land \neg(a = b) \land \neg (b = c) \land \neg (a = c))$
\\
\item
``there exists at most three values of $x$ which makes $\varphi(x)$ true" \\
$\forall x(\exists a \exists b \exists c(\varphi(a) \land \varphi(b) \land \varphi(c)) \to ((\neg(x = a) \land \neg(x = b) \land \neg(x = c)) \to \neg \varphi(x)))$
\\
\item
``for all but finitely many values of $x$ it holds that $\varphi(x)$ is true" \\
$\exists x \forall y(\neg \varphi(y) \to y \leq x)$
\\
\item
``there are infinitely many values of $x$ for which $\varphi(x)$ is true" \\
$\forall x( \varphi(x) \to \exists y(x \leq y \land \neg(x = y) \land \varphi(y)))$
\end{enumerate}
\end{document}

